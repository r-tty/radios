% The Radiant Operating System (RadiOS)
% Copyright (c) 2003 RET & COM Research.
% This file is a LaTeX document.

\section{RadiOS microkernel (rmk)}

\subsection{Overview}

\subsection{Kernel initialization}

\subsection{Page primitives}

The purpose of page primitives is to handle lowest-level memory allocation
and mapping requests. These routines are in file \textit{pages.nasm}.
\\
\textit{PG\_Init} -- initialization of page bitmap. Information about free
and allocated pages is stored by the kernel in a binary map. If a bit in
this map is set, corresponding page is free, otherwise it is allocated.
This approach allows to perform fast and efficient search of free page using
\texttt{BSF} operation.
\\
\textit{PG\_StartPaging} -- creating root page tables and enabling the page
translation. This routine first finds all free pages and marks them in the
bitmap (page is considered free if it doesn't belong to any boot module,
kernel, HMA, special regions (like memory-mapped IO or device memory) or
page bitmap itself). Then it builds a kernel page directory and allocates
enough page tables to provide 1:1 mapping of available physical memory.
Finally, \texttt{CR3} (\texttt{PDBR}) register is initialized with an address
of kernel page directory and page translation is enabled.
\\
\textit{PG\_Alloc} -- page allocation. This routine just scans for a free
bit in the bitmap and returns the address of page found (or ERR\_PG\_NoFreePage
when no memory left in the pool). Allocation is done either in lower memory
(i.e. kernel area) or in upper memory, depending of the value of DL passed
to this routine.
\\
\textit{PG\_Dealloc} -- page deallocation. This routine only sets the
corresponding bit in the page bitmap, thus marking a page as free.
\\
\textit{PG\_AllocDir}
\\
\textit{PG\_AllocAreaTables}
\\

\subsection{Hash functions}

RadiOS microkernel has a centralized mechanism for building and using the
hash tables.
\\
\textit{K\_CreateHashTab}
\\
\textit{K\_HashAdd}
\\
\textit{K\_HashRelease}
\\
\textit{K\_HashLookup}
\\

\subsection{Pools}

\textit{K\_PoolInit}
\textit{K\_PoolAllocChunk}
\textit{K\_PoolFreeChunk}
\textit{K\_PoolChunkNumber}
\textit{K\_PoolChunkAddr}

\subsection{Thread management}

\subsection{Overview of system calls}

There are five groups of system calls in RadiOS:

\begin{itemize}
\item{Channels and connections management}
\item{Message passing primitives}
\item{Synchronization primitives}
\item{Interrupt handling}
\item{Clock services}
\end{itemize}

\subsection{Channels and connections management}

\subsection{Message passing primitives}

\subsection{Synchronization primitives}

\subsection{Exception and interrupt handling}

\subsection{Clock services}

\subsection{Task manager private system calls}
